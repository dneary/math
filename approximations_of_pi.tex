\documentclass[10pt,oneside,landscape]{article}
\usepackage{latexsym}
\usepackage{amsmath}
\usepackage{amssymb}
\usepackage{graphicx}
\usepackage{amsthm}

\newtheorem{theorem}{Theorem}[section]
\newtheorem*{problem}{Problem}

\begin{document}

\title{Approximating $\pi$ with an Infinite Series}
\author{Dave Neary}

\maketitle

\section{Introduction}

We can sum the infinite series:
\[f(t) = 1-t^2+t^4-t^6+\cdots = \frac{1}{1+t^2} \]

Integrating across from 0 to $x$:

\[\int_{t=0}^x f(t) dt = \tan^{-1}(x) = x-\frac{x^3}{3}+\frac{x^5}{5} - \cdots\]

And setting $x=1$ (which converges in the integral, but not in the original function):
$\tan^{-1}(1) = \frac{\pi}{4} = 1-\frac{1}{3}+\frac{1}{5}-\frac{1}{5} - \cdots$

\bigskip

This is a nice formula for $\pi$, but is slow to converge. You can use the tan sum formula to
find faster converging solutions:
\[\tan(A+B) = \frac{\tan A + \tan B}{1-\tan A \tan B}\]

Let $A=\tan^{-1} \frac{1}{2}, B=\tan^{-1} \frac{1}{3}$ Then apply the tangent sum forula, and take
inverse tangents on both side:

\begin{align*}
\tan(A+B) &= \frac{\frac{1}{2}+\frac{1}{3}}{1-\frac{1}{2}\cdot\frac{1}{3}} = 1 \\
\tan^{-1}(1) & = \frac{\pi}{4} = \tan^{-1}\frac{1}{2} + \tan^{-1}\frac{1}{3}
\end{align*}

\bigskip

Then:
\[\displaystyle \frac{\pi}{4} = \sum_{i=1}^{\infty} \frac{(-1)^{i+1}}{2i-1} \left( \frac{1}{2^{2i-1}} + \frac{1}{3^{2i-1}}\right)\]

which converges many times faster. You can do this for any other numbers you can find for
$A+B=1-AB$ which gives $(A+1)(B+1) = 2$ such as $\frac{1}{4}$ and $\frac{3}{5}$.
Another example, more complicated to find, is:

\[\frac{\pi}{4} = 4\tan^{-1}\left(\frac{1}{5}\right) - \tan^{-1}\left(\frac{1}{239}\right)\]

which converges to 10 digits of $\pi$ when you take the first 4 terms of the expansion of
$\tan^{-1} \frac{1}{5}$ and the first two terms of $\tan^{-1} \frac{1}{239}$.

\end{document}

