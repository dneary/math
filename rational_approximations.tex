\documentclass{article}
\usepackage{latexsym}
\usepackage{amsmath}
\usepackage{amssymb}
\usepackage{amsthm}
\usepackage{graphicx}

\newtheorem{theorem}{Theorem}[section]
\renewcommand{\implies}{\Rightarrow}
\begin{document}

\title{Rational approximations of real numbers}
\author{Dave Neary}

\maketitle

\section{Introduction}

Our goal is to prove that we can approximate any positive real number arbitrarily closely
with $\frac{2^a}{3^b}$ with positive integers $a,b$. This surprising result will bring us
into the world of infinite and infinitessimal analysis.

This will be a quick tour around some of the wonders of infinity in analysis, and how to
generate arbitrarily accurate rational approximations for any real number.

\section{Continued fractions}

One way of generating arbitrarily close approximations of irrational numbers is by taking
the partial terms (called "convergents") of the continued fraction expression. Generating
a continued fraction for $x$ involves taking the floor of $x$, subtracting it, taking the
floor of the inverse, and repeating the operation to infinity.

An example will help to illustrate. The continued fraction for $\sqrt{2} = 1.414\cdots$ has
a leading term of 1 (the whole number part of $\sqrt{2}$) and a remainder of $\sqrt{2}-1$.
Inverting the remainder, we find that it is $\frac{1}{\sqrt{2}-1} = \sqrt{2}+1$ - giving
our next term to be $\lfloor 1+\sqrt{2}\rfloor = 2$, and a remainder of $\sqrt{2}-1$.
We can repeat this process ad infinitem to generate the continued fraction 
$\left[1;2,2,2,\cdots\right]$.

\textbf{Exercise:} Prove that the continued fraction of $\sqrt{n^2+1}$ in general is 
$\left[n;2n,2n,2n,\cdots\right]$.

We can now generate a convergent set of rational approximations to $\sqrt{2}$ which are
guaranteed to be the best possible rational approximations with a given denominator (or
smaller) by taking the partial sums:
\[ 1+\frac{1}{2} = \frac{3}{2}\]
\[ 1+\frac{1}{2+\frac{1}{2}} = 1+\frac{2}{5} = \frac{7}{5} \]
\[ 1+\frac{1}{2+\frac{1}{2+\frac{1}{2}}} = 1+\frac{5}{12} = \frac{17}{12} \]
\[ 1+\frac{1}{2+\frac{1}{2+\frac{1}{2 + \frac{1}{2}}}} = 1+\frac{12}{29} = \frac{41}{29} \]

In general, if the $n$th approximation is $\frac{p}{q}$, the $(n+1)$th approximation is 
$\frac{p+2q}{p+q}$.

\textbf{Exercise:} Prove that this sequence converges to $\sqrt{2}$. How does it converge?
Is it monotonic decreasing, or does it oscillate around the eventual limit? Does this
converge to $\sqrt{2}$ if we start with a different starting point for $\frac{p}{q}$?

Larger numbers in a continued fraction representation imply faster convergence, smaller
numbers mean slower convergence. The slowest converging continued fraction is thus:
\[\left[1;1,1,1,\cdots\right] = 1+\frac{1}{\left[1;1,1,1,\cdots\right]}\]

We can evaluate this exactly using the quadratic equation:
\[ x = 1+\frac{1}{x} \]
or $x = \varphi = \frac{1+\sqrt{5}}{2}$, the golden ratio.

The start continued fraction for $\pi$ is
$\left[3;7,15,1,292,1,1,1,2,1,3,1,14,2,1,1,2,2,2,2,\cdots\right]$ and its first few
convergents are the well known approximations of $\pi: 3, \frac{22}{7}, \frac{333}{106}, \frac{355}{113}$.
Since the next term is 292, we know that this last approximating is extremely close to the actual
value of $\pi$.

\section{Finding general rational approximations}

When given a number $x$, one naive strategy for finding a rational approximation is to multiply
it by a positive integer $n$, take the whole number closest to the result, and call $\frac{m}{n}$ our
estimate. This approach works, and will converge to $x$ over time, but it is far from efficient. For
example, $\frac{22}{7}$ is a much better estimate of $\pi$ than $\frac{31}{10}$ - by increasing the
denominator to 10, we have made our estimate a lot worse.

We can do better by using an approach described by Dirichlet. Dirichlet's approximation theorem
says that if $x$ is irrational, that we can always find $a,b \in \mathbb{Z}$ such that
$|ax-b|<\frac{1}{n}$ for any $n\in \mathbb{N}$. To prove this, consider $\{ax\} = ax - 
\lfloor ax \rfloor$ for all $a < n$. $0<\{ax\}<1$ for all $a$, by definition. If we partition the
interval $[0,1)$ into $n-1$ buckets $[0,\frac{1}{n}) \cup [\frac{1}{n},\frac{2}{n}) \cup \cdots
\cup [\frac{n-1}{n}, 1)$, we are guaranteed that of the $n+1$ values $\{a_i\}_{i=0}^n$, at
least 2 will land in the same bucket, by the Pigeonhole principle. If these two values are $a_i, a_j$,
then $a_i = \{ix\} = ix - \lfloor ix \rfloor, a_j = \{jx\} = jx - \lfloor jx \rfloor$, and

\[ \frac{1}{n} > |\{ix\}-\{jx\}| = | (ix-\lfloor ix \rfloor) - (jx - \lfloor jx \rfloor) | \]
\[ \frac{1}{n} > | (i-j)x - (\lfloor ix \rfloor - \lfloor jx \rfloor ) | = | ax-b | \]

for an integer $b = \lfloor ix \rfloor - \lfloor jx \rfloor$. What's more, regardless of how close
the estimate is for $n$, we can find a better estimate by choosing $n > |\frac{1}{ax-b}|$, when
$\frac{1}{n}$ is smaller than the smallest fractional difference for $a<n$.

This is a bit abstract, so let's work an example through to see how this works in practice. Let's try
to approximate $\pi$ using this method with increasing values of $a$. With $a=10$, for example:

\begin{center}
	\begin{tabular}{|c|c|}
		\hline
		$i$ & $\{i\pi\}$ \\
		\hline
		$0$ & $0$ \\
		$1$ & $0.14159$ \\
		$2$ & $0.28318$ \\
		$3$ & $0.42277$ \\
		$4$ & $0.56637$ \\
		$5$ & $0.70796$ \\
		$6$ & $0.84955$ \\
		$7$ & $0.99114$ \\
		$8$ & $0.13274$ \\
		$9$ & $0.27433$ \\
		$10$ & $0.41592$ \\
		\hline
	\end{tabular}
\end{center}

Now, since we have 10 buckets, representing the $\frac{1}{10}$ths, we are guaranteed that at
least two of these will be in the same bucket (in fact, we see that many pairs are in the same
bucket). Looking down the list at the $\frac{1}{10}$ths digit, we can easily see that the
entries corresponding to $i=1, i=8$, $i=2, i=9$, and $i=3, i=10$ all land in the same buckets.
In other words:
\[ \frac{1}{10}> |\{8\pi\}-\{1\pi\}| = | 8\pi - 25 - (1\pi - 3)| = |7\pi -22| \]
which gives us the well-known estimate for $\pi$ of $\pi \approx \frac{22}{7}$. 

Given a good estimate $\frac{a}{b}$ for $x$, we can guarantee to find a better estimate by 
considering $|bx-a| = y$. Again, considering $\pi$: $|7\pi - 22| \approx 0.00885 > 
\frac{1}{113}$, so we are guaranteed to be able to find a more accurate rational estimate by setting
$a \geq 113$.

This iterative nature of always being able to find a closer approximation by choosing a sufficiently
large value for $n$, is what enables arbitrary approximation of any rational number.

\section{A Pell equation approach}

\textbf{Question:} Source: The Riddler from FiveThirtyEight: 
https://fivethirtyeight.com/features/can-you-find-an-extra-perfect-square/

For some perfect squares, when you remove the last digit, the result is also a perfect square. The
first few squares for which this happens are 16, 49, 169, 256, and 361. Can you find the next 3
numbers for which this is true? How many such numbers are there?

For extra credit, note that $(169,16,1)$ is a triple with this property. Are
there other similar triples?

If we write the larger number as $a^2$, and the smaller number as $b^2$, we get that:
\[ 0 \leq a^2 - 10b^2 < 10 \]

And by looking at $n^2 \pmod{10}$, we can see that $a^2 - 10b^2 \in \{0,1,4,5,6,9\}$.
In addition, if $a^2-10b^2 = 5$ that means that $a=10k+5, a^2 = 100(k^2+k) + 25$, so the last
digit of $b^2$ would have to be 2, which is not an option. And if $a^2-10b^2 = 0$, that means that:
$\frac{a^2}{b^2} = 10 \implies \frac{a}{b} = \sqrt{10}$ - but since $\sqrt{n}$ is irrational unless
$n$ is a perfect square, that is not a possibility either. Thus, we need to find the solutions to
$a^2 - 10b^2 \in \{1,4,6,9\}$.

Let's look first at $a^2-10b^2 = 1$.  We can factor this as a difference of two squares:
\[ a^2 - 10b^2 = (a-\sqrt{10}b)(a+\sqrt{10}b) = 1 \]

Additionally, we can show that if $a,b \in \mathbb{Z}$, then:
\[ (a-\sqrt{10}b)^n = A - \sqrt{10}B \]
\[ (a+\sqrt{10}b)^n = A + \sqrt{10}B \]
for some $A,B \in \mathbb{Z}$ using the binomial expansion. For $n=2$, we get $A=a^2+10b^2, B=2ab$.

From the question, we already have the smallest solution to this equation: $19^2 - 10(6^2) = 1$ - 
and from this we can generate as many solutions as we want by using different values of $n$. Indeed:
\[A = 19^2 + 10\cdot 6^2 = 361 + 360 = 721,  B = 2 \cdot 19 \cdot 6 = 228 \]
\[ 721^2 = 51984, 228^2 = 51984, 721^2 - 10\cdot 228^2 = 1 \]

By defining an appropriate norm on numbers in this set (called formally $\mathbb{Z}[\sqrt{10}]$), 
we can also find all solutions satisfying the question. Define the norm as:
\[ N(a+\sqrt{10}b) = (a+\sqrt{10}b)(a-\sqrt{10}b) = a^2-10b^2 \]
Then:
\begin{align*}
	N((a+\sqrt{10}b)(c+\sqrt{10}d)) &= N((ac+10bd) + \sqrt{10}(bc + ad)) \\
	&= (ac+10bd)^2-10(bc+ad)^2 \\
	&= a^2c^2 - 10b^2c^2 - 10a^2b^2 + 100b^2d^2 \\
	N(a+\sqrt{10}b)N(c+\sqrt{10}d) &= (a^2-10b^2)(c^2-10d^2) \\
	&= a^2c^2 -10b^2c^2 -10 a^2d^2 +100b^2d^2 \\
	&= N((a+\sqrt{10}b)(c+\sqrt{10}d))
\end{align*}

So for $\mathbb{Z}[\sqrt{10}]$, if we have a number with a norm $N(a+\sqrt{10}b) = 4$ then we can
generate infinitely many others with the same norm by multiplying by a number that has a norm of 1
(that is, by any number of the form $(19+6\sqrt{10})^n$). Since $N(2 + 0\sqrt{10}) = 4$, any number
of the form $2(19+6\sqrt{10})^n$ will have $a^2-10b^2 = 4$.

For $a^2 - 10b^2 = 6$, we can similarly generate solutions of the form $(4+\sqrt{10})(19+6\sqrt{10})^n$
and $(16+5\sqrt{10})(19+6\sqrt{10})^n$. And finally, for $a^2 - 10b^2 = 9$, there are several numbers with
$N(a+\sqrt{10}b) = 9$: $3, 7+2\sqrt{10}, 13 + 4\sqrt{10}$, and so we can generate solutions for all of
$3(19 + 6\sqrt{10})^n, (7 + 2\sqrt{10})(19 + 6\sqrt{10})^n, (13 + 4\sqrt{10})(19 + 6\sqrt{10})^n$.

It's worth noting that for all solutions of the form $a^2-10b^2 = 1$:
\[ \frac{a}{b} = \sqrt{10 + \frac{1}{b^2}} \approx \sqrt{10} \]

so the solutions to the question above are also increasingly accurate rational approximations of 
$\sqrt{10}$. As we saw in the section on continued fractions, $\sqrt{10} = [3;6,6,6,\cdots]$ - so we
also can create solutions with convergents of the continued fraction above (taken after an
even number of terms, because it oscillates around $\sqrt{10}$). The first convergent,
$\frac{19}{6}$, corresponds to the smallest solution of $a^2-10b^2=1$. The next convergents are
\[ 3 + \frac{1}{6 + \frac{1}{6 + \frac{1}{6}}} = \frac{721}{228} \]
\[ 3 + \frac{1}{6 + \frac{1}{6 + \frac{1}{6 + \frac{1}{6 + \frac{1}{6}}}}} = \frac{27379}{8658} \]
which correspond to the next two solutions to $a^2-10b^2=1$.

For the extra credit: if $0<a^2-10b^2 <10, 0<10b^2 - (10c)^2 \leq 90$, then adding these,

\[0 < a^2 - (10c)^2 = (a-10c)(a+10c) < 100 \]

and $a > 10c$, giving us:

\[ 20c < a+10c < (a+10c)(a-10c) = a^2 - (10c)^2 < 100 \]
Which means $c < 5$, and $a^2$ must begin with $1,4,9,16$. The only option for $b$ with this
constraint is $169$, and there are no other solutions.

\section{Approximating any positive real number with $\frac{2^a}{3^b}, a,b \in \mathbb{Z}^+$}

\textbf{Theorem:} Given $\epsilon>0, x \in \mathbb{R}^+$, we can find $a,b \in \mathbb{Z}^+$ such that:
\[ \left|x - \frac{2^a}{3^b} \right| < \epsilon \]

\textbf{Proof:} Consider for any $a,b \in \mathbb{Z}$ that the following 
equations are equivalent:
\begin{align*}
	y &= \frac{2^a}{3^b} \\
	\log_3(y) &= a\log_3(2) - b
\end{align*}
We will use this second form to determine integers that satisfy our desired inequality.

Since $\log_3(2)$ is irrational, we can use Dirichelet's approximation theorem to find
$a,b \in \mathbb{Z}$ such that:
\[ 0 < a\log_3(2) - b < \frac{1}{n} \]
for any $n \in \mathbb{N}$.

Now given a positive real number $x$, we can find an
integer $N$ such that:
\begin{align*}
	N-1 &<& \frac{\log_3(x)}{a\log_3(2) - b} &\leq& N \\
	(N-1)(a\log_3(2) - b) &<& \log_3(x) &\leq& N(a\log_3(2) - b) \\
	\log_3(x) &\leq& N(a\log_3(2) - b) &<& \log_3(x) + \frac{1}{n} \\
	\log_3(x) &\leq& Na\log_3(2) - Nb &<& \log_3(x) + \frac{1}{n} 
\end{align*}


Since $f(x) = 3^x$ is a monotonic increasing bijection, we can say that we can take
the exponential of all values base 3 in this inequality:
\begin{align*}
	3^{\log_3(x)} &<&  3^{Na\log_3(2) - Nb} & &<& 3^{\log_3(x) + \frac{1}{n}} \\
	x &<&  \frac{3^{Na\log_3(2)}}{3^{Nb}}  &&<& 3^{\log_3(x)}3^{\frac{1}{n}} \\
	x &<&  \frac{2^{Na}}{3^{Nb}}  &&<& x3^{\frac{1}{n}} 
\end{align*}
for positive real $x$, an arbitrary natural number $n$, and some integers $a, b, N$.

Finally, given $x, \epsilon > 0 \in \mathbb{R}$, we can find a positive integer $n \in \mathbb{N}$ such that:
\[ 0 < \frac{2^{Na}}{3^{Nb}} - x  < x(3^{\frac{1}{n}} - 1) < \epsilon  \]
since $\lim_{n \to \infty} (3^{\frac{1}{n}} - 1) = 0$.

Working through an example to find $a,b \in \mathbb{Z}$ will make this mechanism clear. Let's say that we want
to approximate $\pi \approx 3.14159$ to an accuracy of $0.01$ with $\frac{2^a}{3^b}$. First, let's find the
accuracy we will need to approximate $\log_3(\pi)$ to get $0.01$ accuracy at the end:
\begin{align*}
	\pi(3^{\frac{1}{n}} - 1) &< \frac{1}{100} \\
	3^{\frac{1}{n}} &< 1 + \frac{1}{100\pi} \\
	\frac{1}{n} &< \log_3(1 + \frac{1}{100\pi}) \\
	n &> 346
\end{align*}

To make calculations easier, let's take $n = 1000$. We will now find $a,b \in \mathbb{Z}$ such that:
\[ \left| a\log_3(2) - b \right| < \frac{1}{1000} \]
With the help of a spreadsheet, I can find that:
\begin{align*}
	\left| \{783\log_3(2)\} - \{298\log_3(2)\}\right| &\approx 0.00093 < \frac{1}{1000} \\
	\left| 783\log_3(2) - \lfloor 783\log_3(2) \rfloor - (298\log_3(2) - \lfloor 298 \log_3(2) \rfloor) \right| &< \frac{1}{1000} \\
	\left| (783 - 298)\log_3(2) - (\lfloor 783\log_3(2) \rfloor - \lfloor 298 \log_3(2) \rfloor) \right| &< \frac{1}{1000} \\
	\left| 485\log_3(2) - (494 - 188) \right| &< \frac{1}{1000} \\
	\left| 485\log_3(2) - 306 \right| &< \frac{1}{1000} 
\end{align*}
In other words, $\left| \log_3(2) - \frac{306}{485} \right| < \frac{1}{485000}$, and $\frac{306}{485}$ is a very good rational
approximation for $\log_3(2)$.

Next, by dividing $\log_3(\pi)$ by $485\log_3(2) - 306$, we find that:
\[ \frac{\log_3(\pi)}{485\log_3(2) - 306} \approx 1119.825 \]

This gives:
\[ \left| 1120(485\log_3(2) - 306) - \log_3(\pi) \right| \approx 0.0001619 < \frac{1}{1000} \]

Multiplying through, and removing absolute value signs since the resulting value is positive
(if it weren't we could just add one more to the product to make it so):
\[ 543200\log_3(2) - 342720 - \log_3(\pi) \approx 0.0001619 < \frac{1}{1000} \]
\[ 543200\log_3(2) - 342720 < \log_3(\pi) + \frac{1}{1000} \]

And finally, exponentiating everything by 3 and subtracting $\pi$ from both sides, according to Wolfram Alpha,
we have our answer to the desired precision:
\[ \left| \frac{2^{543200}}{3^{342720}} - \pi \right| < \pi(3^{\frac{1}{1000}} - 1) \approx 0.00345 < \frac{1}{100} \]

In fact, we are well inside the desired bound of $\frac{1}{100}$: calculating exactly, we find that:
\[ \left| \frac{2^{543200}}{3^{342720}} - \pi \right| \approx 0.000559 < \frac{1}{1788} \]

\end{document}
