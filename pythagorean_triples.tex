\documentclass{article}
\usepackage{latexsym}
\usepackage{amsmath}
\usepackage{amssymb}
\usepackage{graphicx}
\usepackage{amsthm}
\usepackage{tikz}

\begin{document}

\title{Pythagorean triples}
\author{Dave Neary}

\maketitle

\section{Finding all Pythagorean triples}

A Pythagorean triple is a set of three positive integers which 
are the side lengths of a right-angled triangle - that is, by
Pythagoras's Theorem, the triple $(a,b,c)$ satisfies 
$a^2 + b^2 = c^2$. There are many well-known examples of these,
including $(3,4,5), (5,12,13), (7,40,41)$ - and so on. Our challenge
is to find a way by which we can identify all possible Pythagorean
triples.

To do this, we are going to consider lines with a rational slope, through the
point $(-1,0)$ as shown in the figure below.

\begin{tikzpicture}[scale=3,cap=round]

\draw (0,0) circle (1cm);

\draw[->] (-1.2,0) -- (1.2,0) node[right] {$x$};
\draw[->] (0,-1.2) -- (0,1.2) node[above] {$y$};
\draw[xshift=-1 cm] (0pt,1pt) -- (0pt,-1pt) node[below,fill=white] {$-1$};
\draw[xshift=1 cm] (0pt,1pt) -- (0pt,-1pt) node[below,fill=white] {$1$};
\draw[yshift=-1 cm] (1pt,0pt) -- (-1pt,0pt) node[left,fill=white] {$-1$};
\draw[yshift=1 cm] (1pt,0pt) -- (-1pt,0pt) node[left,fill=white] {$1$};

\draw[thick] (-1.4,-0.1) -- (1.2,0.55);
\draw[very thick, blue] (30:0cm) -- node[right=1pt,fill=white] {$\frac{m}{n}$} + (0,0.25);

\filldraw[blue] (0.882,0.47) circle[radius=0.5pt]
node[above=5pt,fill=white] {$(x_1, y_1)$};

\filldraw[blue] (-1,0) circle[radius=0.5pt]
node[above left=4pt,fill=white] {$(-1, 0)$};
\end{tikzpicture}

A line with a rational slope $\frac{m}{n}$ through $(-1,0)$ will intersect 
the $y$-axis at the point $(0,\frac{m}{n})$. It will also intersect the
unit circle $x^2 + y^2 = 1$ in two points: $(-1,0)$ and $(x_1,y_1)$. Since
the unit circle is a quadratic curve, and the slope of the line is
rational, then if one intersection point is a rational point, the second
intersection must also be rational.

We will find its coordinates in terms of the slope $\frac{m}{n}$. In
slope-intercept form, the equation of the line is:

\[ y = \frac{m}{n}(x+1) \]

Substituting this expression for $y$ into the equation for the circle, we
get the intersection points:

\begin{align*}
x^2 + \left(\frac{m}{n}(x+1)\right)^2 &= 1 \\
x^2 + \frac{m^2}{n^2} x^2 + \frac{2m^2}{n^2} x + \frac{m^2}{n^2} - 1 &= 0 \\
\left(\frac{m^2+n^2}{n^2}\right) x^2 + \frac{2m^2}{n^2} x + \frac{m^2-n^2}{n^2} &= 0 \\
\left(x+1\right)\left(\frac{m^2+n^2}{n^2} x + \frac{m^2-n^2}{n^2}\right) &= 0
\end{align*}

So we have intersection points when $x = -1$ and $x = \frac{n^2-m^2}{n^2+m^2}$. 
Substituting this value for $x$ back into the line equation gives:
\begin{align*}
y_1 &= \frac{m}{n}\left(\frac{n^2-m^2}{n^2+m^2} + 1 \right) \\
 &= \frac{m}{n}\left(\frac{2n^2}{n^2+m^2}\right) \\
 &= \frac{2mn}{n^2+m^2}
\end{align*}

So for any rational slope $\frac{m}{n}$ we can find a rational point on the
unit circle $\left(\frac{n^2-m^2}{n^2+m^2}, \frac{2nm}{n^2+m^2}\right)$
which, in turn, gives us a Pythagorean triple, by setting 
$(a,b,c)=(n^2-m^2, 2nm, n^2+m^2)$.

Since $a^2 + b^2 = c^2$ is equivalent to 
$\left(\frac{a}{c}\right)^2 + \left(\frac{b}{c}\right)^2 = 1$, we can always turn a 
Pythagorean triple into a rational point on the unit circle.
And since all lines between $(-1,0)$ and a rational point on the unit circle have
a rational slope, and our construction above works for all rational numbers, every
Pythagorean triple will be of the form above, or an integer multiple of a point of
the form above.


\end{document}

