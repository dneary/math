\documentclass{article}
\usepackage[utf8]{inputenc}
\usepackage{amsfonts}
\usepackage{amsmath}

\title{Lines in the projective plane $\mathbb{P}^2$}
\author{Dave Neary}
\date{June 2021}

\begin{document}

\maketitle

\section{Introduction}

The projective plane $\mathbb{P}^2$ is the set of points on the affine plane, plus
the points at infinity representing different slopes. The addition of the points at
infinity guarantees that any two distinct lines intersect at one point exactly. It can
also be describes by the set of triples $[X; Y; Z]$ not all zero, which are equivalent
up to multiplication by a scalar - $[1; 2; 3]$ and $[2; 4; 6]$ refer to the same point.

For a point $[X;Y;Z] \in \mathbb{P}^2$, if $Z \neq 0$ then this maps to the point in
the Cartesian (or affine) plane $(\frac{X}{Z}, \frac{Y}{Z})$. If $Z=0$, the point
corresponds to a point at infinity (the intersection point of all lines with the
slope $-\frac{X}{Y}$).

\section{Lines in the projective plane}

Common results in Euclidean geometry hold also in the projective plane, with some differences.

\begin{itemize}
    \item Given two distinct points in the projective plane $P_1=[a_1; a_2; a_3],
    P_2=[b_1; b_2; b_3]$ there is a unique line $L: \alpha X + \beta Y + \gamma Z = 0$
    which goes through both points.
    \item Given two distinct lines in the projective plane, they intersect at exactly one point.
\end{itemize}

Recall that if $P_1 = \lambda P_2$ for some scalar factor then $P_1=P_2$ (if it helps, think
of these as equivalence classes like fractions: $\frac{4}{2} = \frac{6}{3}$). So the triples
$(a_1, a_2, a_3), (b_1, b_2, b_3)$ are linearly independent if they are distinct points in $\mathbb{P}^2$.

Then:
\[
\begin{pmatrix}
a_1 & a_2 & a_3 \\
b_1 & b_2 & b_3
\end{pmatrix} \begin{pmatrix}
\alpha \\ \beta \\ \gamma
\end{pmatrix} = \begin{pmatrix}
0 \\ 0
\end{pmatrix} 
\]
has a solution, unique up to multiplication by a scalar, with $[\alpha; \beta; \gamma] \in \mathbb{P}^2$.
As we have seen, a point in $\mathbb{P}^2$ is essentially the same as a line through the
origin in $\mathbb{A}^3$. Given $P=[X;Y;Z]$ we can write that line in parametric form with:
\[ L: \vec{p} = \lambda \begin{pmatrix} X \\ Y \\ Z \end{pmatrix} \]

Given this, you can see that the two linearly independent vectors $P_1, P_2$ define a plane 
through the origin in $\mathbb{A}^3$ defined by:
\[ C: \vec{p} = \lambda_1 \begin{pmatrix} a_1 \\ a_2 \\ a_3 \end{pmatrix} 
+ \lambda_2 \begin{pmatrix} b_1 \\ b_2 \\ b_3 \end{pmatrix}\]

This maps to the line in $\mathbb{P}^2$:
\[ L: \alpha X + \beta Y + \gamma Z = 0 \]

where the vector 
\[P_3 = \begin{pmatrix} \alpha \\ \beta \\ \gamma \end{pmatrix} \]

satisfies the relationships $P_1 \cdot P_3 = 0$ and $P_2 \cdot P_3 = 0$ - that is,
$P_3$ is orthogonal to both $P_1$ and $P_2$. In other words, it is a scalar multiple
of $P_1 \times P_2$.

We can calculate $\alpha, \beta, \gamma$ with:
\[ \begin{vmatrix} \alpha & \beta & \gamma \\
a_1 & a_2 & a_3 \\
b_1 & b_2 & b_3 \end{vmatrix} = 0 \]
or:
\[L: \alpha \begin{vmatrix} a_2 & a_3 \\ b_2 & b_3 \end{vmatrix}
- \beta  \begin{vmatrix} a_1 & a_3 \\ b_1 & b_3 \end{vmatrix}
+ \gamma \begin{vmatrix} a_1 & a_2 \\ b_1 & b_2 \end{vmatrix} = 0\]

Similarly, given two lines $L_1: \alpha_1 X + \beta_1 Y + \gamma_1 Z = 0,
L_2: \alpha_2 X + \beta_2 Y + \gamma_2 Z = 0$, $L_1, L_2$ are linearly independent, and
there is a unique solution $[X_1; Y_1; Z_1]$ (up to multiplication by a scalar) for the system:
\[
\begin{pmatrix}
\alpha_1 & \beta_1 & \gamma_1 \\
\alpha_2 & \beta_2 & \gamma_2
\end{pmatrix} \begin{pmatrix}
X_1 \\ Y_1 \\ Z_1
\end{pmatrix} = \begin{pmatrix}
0 \\ 0
\end{pmatrix}
\]

which means that any two homogeneous lines intersect at exactly one point in $\mathbb{P}^2$.

\end{document}
