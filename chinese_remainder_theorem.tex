\documentclass{article}
\usepackage{latexsym}
\usepackage{amsmath}
\usepackage{amssymb}
\usepackage{graphicx}
\usepackage{amsthm}

\newtheorem{theorem}{Theorem}[section]

\begin{document}
\title{Solving simultaneous modular arithmetic equations}
\author{Dave Neary}

\maketitle

\section{The Chinese Remainder Theorem}

\begin{theorem}
	The Chinese Remainder Theorem states that for $n_1,n_2,\cdots,n_k$ pairwise
	coprime integers greater than 1 (that is, $\gcd(n_i,n_j) = 1, i \neq j$)
	whose product is $N$, and integers $a_1,a_2,\cdots,a_k$ with
	$0\leq a_i < n_i$, there is a unique integer $x$ such that $0 \leq x < N$
	and:
	\begin{align*}
		x &\equiv a_1 \pmod{x_1} \\
		x &\equiv a_2 \pmod{x_2} \\
		& \vdots  \\
		x &\equiv a_k \pmod{x_k}
	\end{align*}
\end{theorem}

\section{Applying the theorem}

We will apply the theorem using a construction method which can also be used to
prove the existence and uniqueness of the number $x$ with the following problem:

\textbf{Question:} What is the smallest positive integer that has remainders of 7, 4, and 3
when divided by 8, 9, and 13 respectively?

Since $\gcd(8,9) = \gcd(8,13) = \gcd(9,13) = 1$ we are guaranteed by the Chinese
Remainder Theorem that there will be an answer between 0 and $8\times9\times13 = 936$.

We can construct the solution with the following algorithm. We are given:

\[(m_1,m_2,m_3) = (8,9,13)\]
\[(r_1,r_2,r_3) = (7,4,3)\]

Define:
\[(M_1,M_2,M_3) = (m_2\times m_3,m_1 \times m_3, m_1 \times m_2) = (117,104,72)\]

We use the extended Euclidean algoritm to find $(N_1,N_2,N_3)$ such that:

\[ 1 = M_i N_i + m_i n_i, i\in \{1,2,3\}\]

Then we can calculate:
\[x = r_1 N_1 M_1 + r_2 N_2 M_2 + r_3 N_3 M_3 \pmod{m_1 \times m_2 \times m_3}\]

And since $M_i \equiv 0 \pmod{m_j}, i \neq j$, we are guaranteed that $x$ will
satisfy each of the congruence relations we want.

For $M_1,m_1$:
\begin{align*}
	117 &= 14 \times 8 + 5 & 5 &= 117 - 14 \times 8 \\
	8   &= 1 \times 5 + 3  & 3 &= 15 \times 8 - 117 \\
	5   &= 1 \times 3 + 2  & 2 &= 2 \times 117 - 29 \times 8 \\
	3   &= 1 \times 2 + 1  & 1 &= 44 \times 8 -3 \times 117
\end{align*}

Which yields $N_1 = -3$.

Similarly, for $M_2, M_3$, we get:
\begin{align*}
	1 &= 2 \times 104 - 23 \times 9 & N_2 &= 2 \\
	1 &= 2 \times 72 - 11 \times 13 & N_3 &= 2 \\
\end{align*}

Then we can calculate:

\begin{align*}
	x &= r_1 N_1 M_1 + r_2 N_2 M_2 + r_3 N_3 M_3 \pmod{m_1 \times m_2 \times m_3} \\
	x &= 7(-3)(117)+4(2)(104)+3(2)(72) \pmod{936} \\
	x &= -1193 \pmod{936} = 679 \pmod{936} 
\end{align*}

And we can find all solutions of these equations in integers by adding or
subtracting multiples of the product of the modulos:

\[ x = 679 + 936k, k \in \mathbb{Z} \]


\end{document}
