\documentclass{article}
\usepackage{latexsym}
\usepackage{amsmath}
\usepackage{amssymb}
\usepackage{graphicx}
\usepackage{amsthm}

\newtheorem{theorem}{Theorem}[section]
\newtheorem*{problem}{Problem}

\begin{document}
\title{Problems with primes}
\author{Dave Neary}

\section{When is $1+p+p^2+p^3+p^4$ prime for prime $p$?}

\begin{problem}
	Given a prime number $p$, when is the sum of all of the positive factors of $p^4$
	a perfect square?
\end{problem}

\begin{proof}
	Since $p$ is a prime number, the only factors of $p^4$ are of the form
	$p^i, i \in \{0,1,2,3,4\}$. The sum of the factors is thus:
	\[ 1+p+p^2+p^3+p^4 = \frac{p^5-1}{p-1} \]
	
	Let's call this sum of powers of $p$ $f(p)$ for convenience. We begin our search
	with some exploration. When $n=2, 3, 5, 7, 11$ we obtain the following values:
	\begin{center}
		\begin{tabular}{ |c|c|c| }
			\hline
			$p$ & $f(p)$ & round($\sqrt{f(p)}$)$^2$ \\
			\hline
		2 & 31 & $6^2=36$ \\
		3 & 121 & $11^2=121$ \\
		5 & 781 & $28^2=784$ \\ 
		7 & 2801 & $53^2=2809$ \\ 
		11 & 16105 & $127^2=16129$ \\ 
		\hline 
		\end{tabular}
	\end{center}

	3 is the only prime that produces a perfect square so far, and it seems like there
	is always a square close above the other odd primes. It's also interesting to note
	that the square close to $f(p)$ is a little more than the square of the square of our
	prime (not surprising given the $p^4$ term which will dominate as $p$ grows larger):
	$28 = 5^2 + 3, 53 = 7^2 + 4, 127=11^2 + 6$.

	Let's see if we can get close to $f(p)$ with the square of a quadratic expression.
	Clearly such a quadratic will be of the form:
	\[ (p^2 + ap + b)^2 = p^4 + 2ap^3 + (a^2+2b)p^2 + 2abp + b^2\]

	It makes sense to let $a=\frac{1}{2}$, then:
	\[(p^2 + \frac{p}{2} + b)^2 = p^4 + p^3 + (\frac{1}{4}+2b)p^2 + bp + b^2 \]

	Given that $p$ is an odd prime, we can turn this into an integer expression by
	setting $b=\frac{1}{2}$: 
	\[ (p^2 + \frac{p+1}{2})^2 = p^4 + p^3 + \frac{5p^2}{4}p^2 + \frac{p}{2} + \frac{1}{4} \]
	\[ = \sum_{i=0}^4 p^i + \frac{1}{4}p^2 - \frac{p}{2} - \frac{3}{4} \]

	So we can write:
	\[ (p^2 + \frac{p+1}{2})^2 = \sum_{i=0}^4 p^i + \frac{1}{4}(p+1)(p-3) \]
	or
	\[ (p^2 + \frac{p+1}{2})^2 = \sum_{i=0}^4 p^i + \frac{1}{4}((p-1)^2-4) \]

	So this is bigger than $\sum_{i=0}^4 p^i$ for all $p>3$ (and, as we saw earlier, is
	equal for $p=3$). If we reduce the square by 1, can we prove that this is always
	smaller than $\sum_{i=0}^4 p^i$?

	\begin{align*}
		(p^2 + \frac{p-1}{2})^2 &= p^4 + p^3 - \frac{3p^2}{4} - \frac{p}{2} + \frac{1}{4}\\
		&= \sum_{i=0}^4 p^i -\frac{7}{4}p^2 - \frac{3}{2}p - \frac{3}{4} \\
		&< \sum_{i=0}^4 p^i
	\end{align*}
	since $p>0$.

	So we have shown that:
	\[(p^2 + \frac{p-1}{2})^2 < \sum_{i=0}^4 p^i \leq (p^2 + \frac{p+1}{2})^2 \] 
	with equality if and only if $\frac{1}{4}(p+1)(p-3) = 0$, or when $p=3$

\end{proof}

\section{When is $p^2 + q^2 + 2017$ a perfect square?}

\begin{problem}
        How many prime numbers $p, q$ make $p^2 + q^2 + 2017$ a perfect square?
\end{problem}

\begin{proof}
	Consider the numbers $p^2,q^2,2017 \pmod{4}$.
	\[2017 \equiv 1 \pmod{4} \]
	If $p$ is odd, $p^2 \equiv 1 \pmod{4}$, and if $p$ is even, $p^2 \equiv 0 \pmod{4}$.

	Then: 
	\[ 2017 + p^2 + q^2 \pmod{4} \in \{1,2,3\} \]
	with the value 1 when both $p,q$ are event, 2 if one of them is odd, and 3 if both
	are odd.

	For any integer $a$, $a^2 \equiv 0 \text{ or } 1 \pmod{4}$ - so both $p,q$ must be
	even. The only even prime number is 2, so the only number we need to check is
	\[ 2017 + 2^2 + 2^2 = 2025 = 45^2 \]

	So the only answer is $(p,q) = (2,2)$.
\end{proof}


\end{document}

