\documentclass{article}
\usepackage{latexsym}
\usepackage{amsmath}
\usepackage{amssymb}
\usepackage{graphicx}
\usepackage{amsthm}

\newtheorem{theorem}{Theorem}[section]
\newtheorem*{problem}{Problem}

\begin{document}
\title{Finding Formulae for Power Sums}
\author{Dave Neary}

\maketitle

\section{Introduction}

One of the first formulae you learn when starting to study sequences and series is the
formula for the sum of the first $n$ positive integers. This 
series has been made famous by an anecdote that when Gauss was a school boy, his
teacher, wanting to keep the class busy for a few minutes while they stepped out,
told the class to calculate the sum of the positive whole numbers to 100. Before he
had reached the door, a young Gauss had shouted out the answer.

The "trick" that Gauss used was to realize that by inverting the order of the series
and aligning terms, then adding the two lines, he could simplify things:

\begin{alignat*}{6}
	S &=& 1   &+& 2  &+& \cdots &+& 99 &+& 100 \\
	 S &=& 100 &+& 99 &+& \cdots &+& 2  &+& 1 \\
	2S &=& 101 &+& 101 &+& \cdots &+& 101 &+& 101 \\
\end{alignat*}

In other words, $2S = 100(101), S=5050$, and in general, $S(n) = \frac{(n)(n+1)}{2}$.

The challenge of deriving a closed formula for the powers of integers is an
interesting one, which has resulted in some beautiful mathematics. I will present
one way to prove a given formula using induction, and three different ways to derive
the formula for:

\[ S_k(n) = \sum_{i=1}^{n} i^k \]

These three methods will use the Binomial theorem, a combinatorical identity, and
exponential generating functions.

\section{Proof by induction}

Can we prove, given a formula for $S_k(n)$, that it is true for all $n$? One method
to do so is to use induction. The inductive principle says that if something is true
of the first member of a list, and whenever it is true of a member of the list,
it is also true for the next member of the list, then it is true for every member.

The inductive step starts with the assumption that the condition is true for the
$n^\text{th}$ element of the list, and then starting from that assumption, proving
that the condition must also be true for the next. This can be a little confusing at
first, so let's see it in action.

\begin{problem} Prove that for all $n \in \mathbb{N}$:
	\[ \sum_{i=1}^{n} i^2 = \frac{(n)(n+1)(2n+1)}{6} \]
\end{problem}

\begin{proof}
	If $n = 1$, then $1^2 = \frac{(1)(2)(3)}{6} = 1$, so for the base case, the formula is valid.

	Now assume it is true for $n=k$ -- that is:
	\[ \sum_{i=1}^{k} i^2 = \frac{(k)(k+1)(2k+1)}{6} \]

	Then for $n=k+1$:
	\begin{align*}
		\sum_{i=1}^{k+1} i^2 &= \sum_{i=1}^{k} i^2 + (k+1)^2 \\
		 &= \frac{(k)(k+1)(2k+1)}{6} + (k+1)^2 \\
		 &= \left(k+1\right)\left(\frac{(k)(2k+1) + 6(k+1)}{6}\right) \\
		 &= \left(k+1\right)\left(\frac{2k^2 +7k +6}{6}\right) \\
		 &= \left(k+1\right)\left(\frac{(k+2)(2k+3)}{6}\right) \\
		 &= \frac{(k+1)((k+1)+1)(2(k+1)+1)}{6}
	\end{align*}
	
	Since we have shown that if the formula is valid for $n=k$, then it must also
	be valid for $n=k+1$, and since we have further shown that it is valid for $n=1$,
	therefore, by induction, it is true for all $n \in \mathbb{N}$.
\end{proof}

\section{Binomial expansion}

The Binomial theorem states that:
\[ (x + y)^n = \sum_{i=0}^{n} \binom{n}{i}x^{n-i}y^{i} \]
where $\binom{n}{i} = \frac{(n)(n-1)(\cdots)(n-i+1)}{i!}$ for 
$n \in \mathbb{R}, i \in \mathbb{N}_0$, where $i! = i\times(i-1)!$ and $0!=1$ is the
factorial function. For positive integer values of $n$, we can write this as $\frac{n!}{k!(n-k)!}$.

For example:
\begin{align*}
	(x+y)^5 &= \binom{5}{0}x^5 + \binom{5}{1}x^4y + \binom{5}{2}x^3y^2 + 
	\binom{5}{3}x^2y^3 + \binom{5}{4}xy^4 + \binom{5}{5}y^5 \\
	 &= x^5 + 5x^4y + 10x^3y^2 + 10x^2y^3 + 5xy^4 + y^5
\end{align*}

Unsurprisingly, since $(x+y)^n = (y+x)^n$, the binomial coefficients are symmetrical.
That is, $\binom{n}{k} = \binom{n}{n-k}$.

To find a formula for the sum of the squares of the sequence of integers, we will use the
Binomial expansion of the next power up. That is:
\[(n+1)^3 = n^3 + 3n^2 + 3n + 1 \]

This gives us an expression for the difference between consecutive cubes, which we can add for
all integers less than $n$:

\begin{alignat*}{5}
	(n+1)^3 &-& n^3     &=& 3n^2     &+& 3n     &+& 1 \\
	n^3     &-& (n-1)^3 &=& 3(n-1)^2 &+& 3(n-1) &+& 1 \\
	      &\vdots&      &=&       &\vdots&  &\vdots& \\
	3^3     &-& 2^3     &=& 3(2^2)   &+& 3(2) &+& 1 \\
	2^3     &-& 1^3     &=& 3(1^2)   &+& 3(1) &+& 1 \\
\end{alignat*}

Now if we add the left-hand side, we see that a number of terms cancel out, and we can replace
$\sum_{i=1}^{n} i$ with the formula we derived earlier to get a closed formula for
$\sum_{i=1}^{n} i^2$.

\begin{align*}
	3\sum_{i=1}^{n} i^2 + 3\sum_{i=1}^{n} i + n &= (n+1)^3 - 1^3 \\
	3\sum_{i=1}^{n} i^2 &= n^3 +3n^2+3n - \frac{3(n)(n+1)}{2} - n \\
	3\sum_{i=1}^{n} i^2 &= \frac{1}{2}\left(2n^3 +6n^2+6n - 3n^2 -3n - 2n\right) \\
	\sum_{i=1}^{n} i^2  &= \frac{1}{6}\left(2n^3 +3n^2+n\right) \\
	\sum_{i=1}^{n} i^2  &= \frac{1}{6}\left((n)(n+1)(2n+1)\right) 
\end{align*} 

We can use this method for higher exponents $k$ and, as long as we have formulae for all 
of the lower exponents, we can always generate an explicit formula using this method for
$\sum_{i=1}^{n} i^k$.

\section{Using a Binomial identity}

\subsection{Pascal's Triangle}

The second method is closely related to the first, but avoids the requirement to derive
a formula for all of the lower order exponents. We will use a binomial identity which
we can derive from the well-known property of binomial coefficients which gives us
Pascal's triangle, namely:
\[\binom{n}{k} + \binom{n}{k+1} = \binom{n+1}{k+1} \]

We can prove this identity as follows:

\begin{align*}
	\binom{n}{k} + \binom{n}{k+1} &= \frac{n!}{k!(n-k)!} + \frac{n!}{(k+1)!(n-k-1)!} \\
	&= \frac{n!}{k!(n-k-1)!}\left(\frac{1}{n-k} + \frac{1}{k+1}\right) \\
	&= \frac{n!}{k!(n-k-1)!}\left(\frac{(k+1)+(n-k)}{(n-k)(k+1)}\right) \\
	&= \frac{(n+1)!}{(k+1)!((n+1)-(k+1))!} \\
	&= \binom{n+1}{k+1}
\end{align*}

For those who have not heard of it, Pascal's Triangle is an arrangement of the
binomial coefficients in a triangle, and each term in the triangle is the sum of the two
terms directly above it. The first five rows are as follows:

\begin{tabular}{rccccccccc}
$n=0$:&    &    &    &    &  1\\\noalign{\smallskip\smallskip}
$n=1$:&    &    &    &  1 &    &  1\\\noalign{\smallskip\smallskip}
$n=2$:&    &    &  1 &    &  2 &    &  1\\\noalign{\smallskip\smallskip}
$n=3$:&    &  1 &    &  3 &    &  3 &    &  1\\\noalign{\smallskip\smallskip}
$n=4$:&  1 &    &  4 &    &  6 &    &  4 &    &  1\\\noalign{\smallskip\smallskip}
\end{tabular}

Each line represents the coefficients of a binomial expansion. For example, the line $n=3$
corresponds to $\binom{3}{0}, \binom{3}{1}, \binom{3}{2}, \binom{3}{3}$.

By examining Pascal's Triangle (or repeatedly applying the identity above), you can generate
some interesting identities. The one we are interested in is:
\[ \sum_{i=1}^{n} \binom{i}{k} = \binom{n+1}{k+1} \]

This does require some clarification for the value of $\binom{n}{k}$ when $k>n$. In this case, we use the formula for the binomial coefficient we introduced originally:
\[ \binom{n}{k} = \frac{(n)(n-1)(\cdots)(n-k+1)}{k!} \]

For $n<k$ this will yield $\binom{n}{k} = 0$ - and in the context of an interpretation of
this meaning the number of ways we can "choose $k$ items from $n$ options", this makes sense, as it is impossible to choose $k$ items from fewer than $k$ choices.

This identity corresponds to adding up all of the numbers on the diagonal one over from the
number of interest in Pascal's Triangle. For example:
\[ \binom{5}{3} = \binom{4}{2} + \binom{3}{2} + \binom{2}{2} = 6 + 3 + 1 \]

We can justify this by repeatedly applying the binomial identity:
\begin{align*}
	\binom{n+1}{k+1} &= \binom{n}{k} + \binom{n}{k+1} \\
	&= \binom{n}{k} + \binom{n-1}{k} + \binom{n-1}{k+1} \\
	&= \binom{n}{k} + \binom{n-1}{k} + \binom{n-2}{k} + \binom{n-2}{k+1} \\
	&\vdots \\
	&= \sum_{i=1}^{n} \binom{i}{k} 
\end{align*}

\subsection{Applying the Identity}

$\binom{n}{k}$ is a polynomial in $n$ of order $k$. For example:
\[ \binom{n}{3} = \frac{1}{6}(n)(n-1)(n-2) = \frac{1}{6}(n^3-3n^2+2n) \]

We can express $n^k$ as a linear combination of $\binom{n}{i}$ for $i\leq k$. For $n=3$,
to continue with that example, we have that:
\begin{align*}
	n^3 &= 6\binom{n}{3} +3n^2-2n \\
	&= 6\binom{n}{3} + 6 \binom{n}{2} +\binom{n}{1}
\end{align*}

In this case, we can now apply this identity along with the one above to get a formula
for the sum of cubes:

\begin{align*}
	\sum_{i=1}^{n} i^3 &= 6\sum_{i=1}^{n} \binom{i}{3} + 6\sum_{i=1}^{n} \binom{i}{2}
	+ \sum_{i=1}^{n} \binom{i}{1} \\
	&= 6\binom{n+1}{4} +6\binom{n+1}{3} + \binom{n+1}{2} \\
	&= 6\frac{(n+1)(n)(n-1)(n-2)}{24} +6 \frac{(n+1)(n)(n-1)}{6} + \frac{(n+1)(n)}{2} \\
	&= (n+1)(n) \frac{(n-1)(n-2) + 4(n-1) + 2}{4} \\
	&= \left(\frac{n(n+1)}{2}\right)^2
\end{align*}

Again, we can use this method for arbitrary values of $k$.

It is straightforward to calculate the values of the coefficients of $\binom{n}{i}$ in
the linear combination when we realize that the equation must be true for all values of
$n$, and (as we saw earlier) $\binom{n}{k} = 0$ if $n<k$. So for example, if we wanted to 
calculate the sum $\sum_{i=1}^{n} i^4$, we could use:
\[ n^4 = a_1\binom{n}{1} + a_2\binom{n}{2} + a_3 \binom{n}{3} + a_4 \binom{n}{4} \]
Setting $n=1$, we get the equation:
\[ 1^4 = a_1\binom{1}{1} \implies a_1 = 1 \]

Now setting $n=2$:
\[ 2^4 = a_1\binom{2}{1} + a_2\binom{2}{2} \implies a_2 = 16-2 = 14 \]

With $n=3$:
\[ 3^4 = a_1\binom{3}{1} + a_2\binom{3}{2} + a_3\binom{3}{3}\]
\[\implies a_3 = 81 -3(14) - 3(1)  = 36 \]

Finally, for $n=4$:
\[ 4^4 = a_1\binom{4}{1} + a_2\binom{4}{2} + a_3\binom{4}{3} + a_4\binom{4}{4}\]
\[\implies a_3 = 256 -4(36) - 6(14) - 4(1))  = 24\]

So we have:
\[ n^4 = \binom{n}{1} + 14\binom{n}{2} + 36 \binom{n}{3} + 24 \binom{n}{4} \]

And:
\begin{align*}
	\sum_{i=1}^{n} i^4 &= \binom{n+1}{2} + 14\binom{n+1}{3} + 36 \binom{n+1}{4}
	                      + 24 \binom{n+1}{5} \\
			   &= \frac{1}{30} (n)(n + 1)(2n + 1)(3n^2 +3n - 1)
\end{align*}

\section{Exponential Generating Functions}

I think my favourite method for deriving these formulae is to use generating functions.
This feels the most like magic, and it has a deep connection to the Bernoulli numbers,
which arise often in number theory.

A Generating Function is a polynomial representation of a sequence $\{a_n\}$ such that
$a_i$ is the coefficient of $x^i$ for each $i \geq 0$. For example, the sequence 
$\{1, \frac{1}{2}, \frac{1}{4}, \frac{1}{8}, \cdots \}$ corresponds to the polynomial
\[ P(x) = 1 + \frac{x}{2} + \frac{x^2}{4} + \frac{x^3}{8} + \cdots \]

We have a lot of tools for manipulating generating functions, as long as they satisfy
certain constraints. For example, we can take term-wise derivatives or integrals of
the terms to generate new identities or to calculate complicated infinite sums.

In the case of $P(x)$, you might recognize it as a geometric series. If it is convergent,
we can manipulate it as follows:

\[ \frac{x}{2} P(x) = \frac{x}{2} + \frac{x^2}{4} + \frac{x^3}{8} + \cdots \]
\[ P(x) - \frac{x}{2} P(x) = 1 \]
\[ P(x) = \frac{1}{1-\frac{x}{2}} \]

An exponential generating function is similar, but with one difference. Given the sequence
$\{a_n\}$, the generating function is:
\[ P(x) = a_0 + \frac{a_1x}{1!} + \frac{a_2x^2}{2!} + \frac{a_3x^3}{3!} + \cdots \]

That is, each term is of the form $\frac{a_nx^n}{n!}$. What this gains us is that we are
guaranteed convergence for any sequence which does not grow exponentially. It also gives
us the ability to manipulate sequences using exponential rules.

We can use exponential power series to generate formulae for the sum of $i^k$ for all
values of $k$ simultaneously. Define
\[ S_k(n) = \sum_{i=1}^{n} i^k \]

and consider the corresponding exponential generating function:

\begin{align*}
	P(x) &= S_0(n) + \frac{S_1(n)x^1}{1!} + \frac{S_2(n)x^2}{2!} + \cdots 
	&  & \text{By definintion} \\
	&= \sum_{i=0}^{\infty} \frac{S_i(n)x^i}{i!} &  & \text{Using summation notation} \\
	&= \sum_{i=0}^{n} \sum_{j=1}^{\infty} \frac{j^ix^i}{i!} &  & 
	\text{Breaking apart and distributing } S_i(n) \\
	&= \sum_{j=1}^{n} \sum_{i=0}^{\infty} \frac{j^ix^i}{i!} &  & 
	\text{Swapping order of summation} \\
	&= \sum_{j=1}^{n} e^{jx} &  & \text{Definition of the exponential function} \\
	&= \frac{e^{i(n+1)x}-e^x}{e^{x}-1} &  & \text{Summing a geometric series}
\end{align*}

Expanding $e^x$ and $e^{(n+1)x}$, grouping and cross-multiplying, we get:

\[ \left( \sum_{i=1}^{\infty} \frac{x^i}{i!} \right) 
	\left( \sum_{i=0}^{\infty} \frac{S_i(n)x^i}{i!}\right) = 
	\sum_{i=1}^{\infty} \frac{((n+1)^i - 1)x^i}{i!} \]

\begin{multline*}
	\left( x+\frac{1}{2!}x^2 + \frac{1}{3!}x^3 + \cdots \right)
	\left( S_0(n) + \frac{S_1(n)}{1!}x + \frac{S_2(n)}{2!}x^2 + \cdots \right) = \\
	\left( ((n+1) - 1)x+\frac{((n+1)^2-1)}{2!}x^2 + \frac{(n+1)^3-1)}{3!}x^3 + \cdots \right)
\end{multline*}

By equating the coefficients of $x^{i+1}$ on each side of this equation, we obtain a formula
for $S_i(n)$.

\begin{align*}
	S_0(n) &= n \\
	S_1(n) + \frac{S_0(n)}{2!} &= \frac{1}{2!}((n+1)^2 -1) \\
	\frac{S_2(n)}{2!} + \frac{S_1(n)}{2!} + \frac{S_0(n)}{3!} 
	&= \frac{1}{3!}\left( (n+1)^3 - 1 \right) \\
	\frac{S_3(n)}{3!} + \frac{S_2(n)}{2!2!} + \frac{S_1(n)}{3!} + \frac{S_0(n)}{4!} 
	&= \frac{1}{4!}\left( (n+1)^4 - 1 \right) \\
\end{align*}

This gives the familiar formulae for $S_i(n)$, and the ability to further expand to
calculate $S_4(n)$ and beyond with a recursion formula.

\begin{align*}
	S_1(n) &= \frac{1}{2}n^2 + \frac{1}{2}n \\
	S_2(n) &= \frac{1}{3}n^3 + \frac{1}{2}n^2 + \frac{1}{6}n \\
	S_3(n) &= \frac{1}{4}n^4 + \frac{1}{2}n^3 + \frac{1}{4}n^2 \\
	S_4(n) &= \frac{1}{5}n^5 + \frac{1}{2}n^4 + \frac{1}{3}n^3 - \frac{1}{30}n 
\end{align*}

In general:

\[ \sum_{i=0}^{k} \frac{S_i(n)}{(k - i + 1)!i!} = \frac{(n+1)^{k+1} - 1}{(k+1)!} \]


\end{document}
