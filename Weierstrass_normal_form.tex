\documentclass{article}
\usepackage[utf8]{inputenc}
\usepackage{amsmath}
\usepackage{amsfonts}


\title{Weierstrass normal form}
\author{Dave Neary}
\date{May 2021}

\begin{document}

\maketitle

\section{Weierstrass normal form for cubic equations}

Given a general cubic equation
\[ ax^3 +bx^2y +cxy^2 + dy^3 +ex^2 + fxy +gy^2 + hx + iy +j = 0 \]
we can perform a set of substitutions to convert the equation to a form called Weierstrass normal form:
\[ Y^2 = X^3 + AX^2 + BX + C \]
or, equivalently:
\[ Y^2 = X^3 + \alpha X + \beta \]

In particular, for the cubic equation $C$:
\[u^3+v^3=\alpha \]
we can use the substitutions
\[x = \frac{12\alpha}{u+v}, y = 36\alpha\frac{u-v}{u+v} \]
to convert this equation to the form $C'$:
\[ y^2 = x^3 - 432 \alpha \]

This process can be inverted, so from any point $(x,y) \in C'$ we can generate a point $(u,v) \in C$ by the transform:
\[u = \frac{36\alpha +y}{6x}, v = \frac{36\alpha - y}{6x} \]

By this means, if we find rational points on either curve, we can generate a rational point on the other by this bijection.

This transformation might seem like magic, but we can work through this process if we have a rational point on the homogeneous projective form of original curve. To get to this form, we replace $u = \frac{U}{W}, v = \frac{V}{W}$ and multiply across by $W^3$ to give:
\[ U^3 + V^3 - \alpha W^3 = 0 \]
The point in $\mathbb{P}^2$ $P = [-1; 1; 0]$ is on the curve $C$. Since $W=0$ this corresponds to a point at infinity - a tangent point of the curve. The tangent at the point $P$ is given by the line:
\[ U \frac{\partial C}{\partial U} (P) + V \frac{\partial C}{\partial V} (P) + W \frac{\partial C}{\partial W} (P) = 0 \]

\[ \frac{\partial C}{\partial V} = 3V^2, \frac{\partial C}{\partial U} = 3U^2, \frac{\partial C}{\partial W} = 3\alpha W^2 \]

So at the point $P = [-1;1;0]$ the tangent line in $\mathbb{P}^2$ is $3U + 3V = 0$ (and we can divide out the common factor of 3).

Examining the curve $u^3+v^3=\alpha$ we can see that this does not intersect at all in the
real numbers, and has a triple root at the point at infinity. We will use this line as our
$Z=0$ axis after our transformation.

We will take $U-V=0$ as our $X$ axis, motivated by the fact that this is a line of symmetry
of our curve, and finally we will take $U+V-W = 0$ as our $Y$ axis, motivated by the fact
that it intersects $Z=0$ at $P$, and is helpfully orthogonal to the $X$ axis, so we should
avoid any pesky $XY$ terms after transformation.

Putting this together, our transformation from $(U,V,W)$ space to $(X,Y,Z)$ space will be:
\[ 
\begin{pmatrix} 1 & -1 & 0 \\ 1 & 1 & -1 \\ 1 & 1 & 0 \end{pmatrix}
\begin{pmatrix} U \\ V \\ W \end{pmatrix} = 
\begin{pmatrix} X \\ Y \\ Z \end{pmatrix} \]

This matrix is, by design, invertible. Its inverse is:

\[ A^{-1} = \frac{1}{2} \begin{pmatrix} 1 & 0 & 1 \\ -1 & 0 & 1 \\ 0 & -2 & 2 \end{pmatrix} \]
This gives us an affine transformation which allows us to go from $(X,Y,Z)$ space to $(U,V,W)$ space as follows:
\begin{equation*}
\begin{split}
    U &= \frac{1}{2}(X + Z) \\
    V &= \frac{1}{2}(-X + Z) \\
    W  &= -Y + Z
\end{split}
\end{equation*}

By substituting these equations back into our homogeneous version of $C$, we get:
\[ \left(\frac{1}{2}(X + Z)\right)^3 + \left(\frac{1}{2}(-X + Z)\right)^3
- \alpha\left(-Y + Z\right)^3 = 0 \]

Expanding and cancelling terms, we simplify our original equation to:
\begin{equation*}
\begin{split}
     \frac{1}{8}(6X^2Z + 2Z^3) - \alpha (-Y^3 + 3Y^2Z - 3YZ^2 + Z^3) &= 0 \\
     3X^2Z + Z^3 + 4 \alpha Y^3 - 12 \alpha Y^2Z + 12\alpha YZ^2 - 4\alpha Z^3 &= 0 
\end{split}
\end{equation*}

Replacing $x=\frac{X}{Z}, y=\frac{Y}{Z}$ to dehomogenize, we get:
\[ 3x^2 + 1 + 4 \alpha y^3 - 12 \alpha y^2 + 12\alpha y - 4\alpha = 0 \]
\[ 3x^2 = -4\alpha y^3 + 12\alpha y^2 -12 \alpha y +4 \alpha -1 \]

We want a perfect square term on the left, and a leading perfect cube on the right. We can achieve
this by multiplying both sides by $3^3\cdot4^2\cdot \alpha^2 = 432\alpha^2$:

\[ 3^4 4^2 \alpha^2 x^2 = -3^3 4^3 \alpha^3 y^3 + 3^3 4^3 \alpha^3 (3y^2) - 3^3 4^3 \alpha^3(3 y) +3^3 4^3 \alpha^3 - 432\alpha^2 \]

Substituting $a=36\alpha x, b=-12\alpha(y-1)$ we get:
\[ a^2 = b^3 - 432\alpha^2 \]

The transforms from $(a,b)$ to $(u,v)$ are now straightforward to derive:
\begin{equation*}
	\begin{array}{llll}
		a & = & 36\alpha x & \quad \mathrel{\#} \text{Final substitution above} \\
		& = & 36 \alpha \frac{X}{Z} & \quad \mathrel{\#} x = \frac{X}{Z} \\
		& = & 36 \alpha \frac{U-V}{U+V}  & \quad \mathrel{\#} X = U - V, Z = U + V \\
		& = & 36 \alpha \frac{u-v}{u+v} & \quad \mathrel{\#} u = \frac{U}{W}, v = \frac{V}{W} \\
		b & = & -12 \alpha (y-1) & \quad \mathrel{\#} \text{Final substitution above} \\
		& = & -12 \alpha \left(\frac{Y}{Z} - 1 \right) & \quad \mathrel{\#} y = \frac{Y}{Z} \\
		& = & -12 \alpha \left(\frac{U+V-W}{U+V} - 1 \right) & \quad \mathrel{\#} Y = U+V-W, Z= U + V \\
		& = & -12 \alpha \left(\frac{-W}{U+V}\right) & \quad \mathrel{\#} \text{Simplify } \frac{U+V}{U+V}=1 \\
		& = & \frac{12 \alpha}{u+v} & \quad \mathrel{\#} u = \frac{U}{W}, v = \frac{V}{W} 
	\end{array}
\end{equation*}

The transformation from $(u,v)$ to $(a,b)$ is similar:
\begin{equation*}
	\begin{array}{llll}
		u & = & \frac{U}{W} & \quad \mathrel{\#} \text{Homogenization of original cubic curve} \\
		& = & \frac{\frac{1}{2}(X + Z)}{-Y + Z} & \quad \mathrel{\#} U = \frac{1}{2}(X + Z), W = -Y + Z \\
		& = & \frac{\frac{X}{Z} + 1}{-2\frac{Y}{Z} + 2} & \quad \mathrel{\#} \text{Division by Z in numerator and denominator} \\
		& = & \frac{(x + 1)}{-2(y - 1)} & \quad \mathrel{\#} x = \frac{X}{Z}, y = \frac{Y}{Z} \\
		& = & \frac{(\frac{a}{36\alpha} + 1)}{-2\left(\frac{-b}{12\alpha} \right)} & \quad \mathrel{\#} a = 36 \alpha x, b = -12 \alpha (y-1) \\
		& = & \frac{a + 36\alpha}{6b} & \quad \mathrel{\#} \text{Simplify fraction} \\
		v & = & \frac{V}{W} & \quad \mathrel{\#} \text{Homogenization of original cubic curve} \\
		& = & \frac{\frac{1}{2}(-X + Z)}{-Y + Z} & \quad \mathrel{\#} V = \frac{1}{2}(-X + Z), W = -Y + Z \\
		& = & \frac{-X + Z}{-2Y + 2Z} & \quad \mathrel{\#} \text{Simplify fraction} \\
		& = & \frac{-x + 1}{-2(y -1)} & \quad \mathrel{\#} x = \frac{X}{Z}, y = \frac{Y}{Z} \\
		& = & \frac{-\frac{a}{36\alpha} + 1}{-2(\frac{-b}{12\alpha})} & \quad \mathrel{\#} a = 36 \alpha x, b = -12 \alpha (y-1) \\
		& = & \frac{-a + 36\alpha}{6b} & \quad \mathrel{\#} \text{Simplify fraction} 
	\end{array}
\end{equation*}

\end{document}
