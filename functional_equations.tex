\documentclass{article}
\usepackage{latexsym}
\usepackage{amsmath}
\usepackage{amssymb}
\usepackage{graphicx}
\usepackage{amsthm}

\newtheorem{theorem}{Theorem}[section]
\newtheorem*{problem}{Problem}

\begin{document}

\title{Some functional equations}
\author{Dave Neary}

\maketitle

\section{2001 Dutch IMO qualification}

Suppose for all $x,y \in \mathbb{R}$ that we have:
\[ f(x + y) = f(x) + f(y) + xy \]
and $f(4) = 10$. Find $f(2001)$.

Setting $x=y=2$:
\[ f(4) = 10 = 2f(2) + 4 \implies f(2) = 3\]

Setting $x=y=1$:
\[ f(2) = 3 = 2f(r12) + 1 \implies f(1) = 1\]

Setting $y=0$:
\[ f(x) = f(x) + f(0) \implies f(0) = 0 \]

Finally, setting $y=1$:
\[ f(x+1) = f(x) + f(1) + x \]

This is already enough for us to calculate $f(n)$ for all positive integers:
\begin{align}
	f(n) - f(n-1) & = n \\
	& \vdots \\
	f(1)-f(0) &= 1
\end{align}
Adding LHS and RHS we get:
\[ f(n) = \sum_{i=1}^{n} i = \frac{1}{2}(n)(n+1) \]

and $f(2001) = 2003001$.

We can easily extend to show that the formula applies to negative integers:
\[f(n + (-n)) f(0) = 0 = f(n) + f(-n) - n^2 \]
\[ \implies f(-n) = n^2 - \frac{1}{2}(n)(n+1) = n^2 - \frac{n^2}{2} - \frac{n}{2} \]
\[ \implies f(-n) = \frac{1}{2}(-n)(-n+1) \]


The question remains, however: what class of functions satisfies the functional relationship in
general, outside of integers?

Since $f(nx) = f((n-1)x) + f(x) + (n-1)x^2$, we can show by simple iteration that:
\[ f(nx) = nf(x) + x^2((n-1)+(n-2)+\cdots + 1) = nf(x) + \frac{(n)(n-1)}{2} x^2 \]

From above, we know that $f(n) = \frac{1}{2}(n)(n+1)$ for $n\in \mathbb{Z}$, so
for $x=\frac{1}{m}, n=m, m \in \mathbb{Z}/{0}$:
\begin{align}
	f(1) &= f(m\times\frac{1}{m}) \\
	     &= mf(\frac{1}{m}) + \frac{m^2-m}{2m^2} \\
	     &= 1 \\
	f(\frac{1}{m}) &= \frac{1}{m}(1- \frac{1}{2}(1-\frac{1}{m})) \\
	               &= \frac{1}{2}(\frac{1}{m})(\frac{1}{m} + 1)
\end{align}

We can now generalize to all rational numbers:
\begin{align}
	f(\frac{n}{m}) &= nf(\frac{1}{m}) + \frac{n^2-n}{2m^2} \\
	& = \frac{n}{2}(\frac{1}{m})(\frac{1}{m} + 1) + \frac{1}{2}((\frac{n}{m})^2-\frac{n}{m^2}) \\
	& = \frac{1}{2}(\frac{n}{m})(\frac{n}{m} + 1)
\end{align}

But does this mean that $f(x) = \frac{1}{2}(x)(x+1)$ for all $x\in \mathbb{R}$? Bizarrely enough,
the answer is no. However, if $f(x)$ is defined to be continuous, then this is true. To see this,
let's consider an arbitrary irrational number $x$. We can define a sequence of rational numbers 
$\{a_n\}$ with $10^nx > 10^n a_n > 10^n x - 1$ for all $n$ by defining 
$a_n = \frac{1}{10^n}\lfloor 10^nx \rfloor$. This set of rational numbers will converge to $x$, and the
set $\{f(a_n)\} = \{\frac{1}{2}(a_n)(a_n+1)\}$ also converges to $f(x) = \frac{1}{2}(a_n)(a_n+1)$
if $f$ is continuous.

However, if $f$ is not continuous, we can construct infinitely many $f(x)$ which satisfy the
conditions. For any irrational number, we can arbitrarily define $f(x) = c$, and then
\[ f(qx) &= qc + \frac{1}{2}(q)(q-1)(x^2) \]

And since $qx$ will never be a rational number, this is linearly independent to the rational
numbers we already have above.

\end{document}
