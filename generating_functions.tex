\documentclass{article}
\usepackage[utf8]{inputenc}
\usepackage{amsmath}
\usepackage{amsfonts}

\title{Generating functions and counting problems}
\author{Dave Neary}
\date{October 2021}

\begin{document}

\maketitle

\section{Problem}

How many solutions are there to the equation $n = a + b + c + d$, where $a \leq b \leq c \leq d$ and $a, b, c, d, n \in \mathbb{N}$?

\section{Solution}

The natural numbers $\mathbb{N} = \{1,2,3,\cdots\}$ — that is, the positive integers (not 0). It will be easier to work on this problem if our solutions are in the non-negative integer, including 0.

We can rewrite the equation as:
\[ x_1= a'+b'+c'+d' \]
where: $x_1=x-4, a'=a-1, b'=b-1, c' = c-1, d' = d-1$, and the inequality for $a,b,c,d$ still holds for $a',b',c',d'$.

We can go one step further, and remove this inequality, by focusing on the
differences between the variables. Since we know that $a'\leq b'\leq c'\leq d'$ we can rewrite 
$a_1 = a', b_1 = b' - a', c_1 = c' - b', d_1 = d' - c'$, and substituting these in to the equation, 
we get:

\[ x_1= 4a_1 + 3b_1 + 2c_1 + d_1 \]

where each of $a_1,b_1,c_1,d_1 \geq 0$.

There's a nice method of calculating this using generating functions. Consider:

\[P(x) = (1+x^4+x^8+\cdots)(1+x^3+x^6+\cdots)(1+x^2+x^4+\cdots)(1+x+x^2+\cdots)\]

This expands to an infinite series:
\[P(x) = a_0 + a_1x + a_2x^2 + a_3x^3 + \cdots\]

where each coefficient corresponds to the number of solutions for $x_1 = n, x = n+4$.

To get the coefficient in the expansion for $x^{12}$ for example, we need to look at all
the ways that you can combine multiples of 1,2,3, and 4 to add to 12. We can quickly find
that the possible solutions are:

\[(a_1,b_1,c_1,d_1) \in \{(3,0,0,0),(2,1,0,1),(2,0,2,0),(2,0,1,2),(2,0,0,4), \cdots\} \]

where the solution $(2,1,0,1)$ for example corresponds to taking an $x^8$ term from
$(1+x^4+x^8+\cdots)$, an $x^3$ from $(1+x^3+x^6+\cdots)$, a 1 from $(1+x^2+x^4+\cdots)$,
and an $x$ from $(1+x+x^2+\cdots)$.

Then we can sum these geometric series to get:

\[P(x) = \frac{1}{(1-x^4)(1-x^3)(1-x^2)(1-x)} \]

We can use partial fraction decomposition on this to get a closed form for $a_n$.

Our denominator can be factored:
\begin{align*}
 P(x) &= \frac{1}{(1-x^4)(1-x^3)(1-x^2)(1-x)} \\
      &= \frac{1}{(1-x)^4(1+x)^2(1+x^2)(1+x+x^2)} \\
      &= \frac{1}{(1-x)^4(1+x)^2(1+ix)(1-ix)(1-\omega x)(1 - \omega^2 x)}
\end{align*}

where $\omega = \frac{1}{2}(-1+\sqrt{3}i)$, a primitive cube root of unity.

Then we can write this as:
\begin{multline*}
    P(x) = \frac{A_1}{1 - x} + \frac{A_2}{(1 - x)^2} + \frac{A_3}{(1 - x)^3} + \frac{A_4}{(1 - x)^4} \\
    + \frac{A_5}{1 + x} + \frac{A_6}{(1 + x)^2} \\
    + \frac{A_7}{1 + ix} + \frac{A_8}{1 - ix} + \frac{A_9}{1 - \omega x} + \frac{A_{10}}{1 - \omega^2 x}
\end{multline*}

Calculating the $A_i$ values is tedious but straightforward using the Heaviside cover-up
method. Clearing denominators, we get:
\begin{align*}
    1 &= A_1(1-x)^3(1+x)^2(1+x^2)(1+x+x^2) + A_2(1-x)^2(1+x)^2(1+x^2)(1+x+x^2) \\
    &+ A_3(1-x)(1+x)^2(1+x^2)(1+x+x^2) + A_4(1+x)^2(1+x^2)(1+x+x^2) \\
    &+ A_5(1-x)^4(1+x)(1+x^2)(1+x+x^2) + A_6(1-x)^4(1+x^2)(1+x+x^2) \\
    &+ A_7(1-x)^4(1+x)^2(1-ix)(1+x+x^2) + A_8(1-x)^4(1+x)^2(1+ix)(1+x+x^2) \\
    &+ A_9(1-x)^4(1+x)^2(1+x^2)(1-\omega^2 x) + A_{10}(1-x)^4(1+x)^2(1+x^2)(1 - \omega x)
\end{align*}

Now we can set $x$ to various values to isolate and calculate the coefficients (since the equation above must hold for all values of $x$).

Setting $x=1, x=-1, x=i, x=-i, x=\omega^2, x=\omega$ in order we get 
\[
A_4 = \frac{1}{24}, A_6 = \frac{1}{32}, A_7 = \frac{1}{16}, A_8 = \frac{1}{16}, A_9 = \frac{1-\omega}{27}, A_{10} = \frac{1-\omega^2}{27}
\]

Then I set $x=2, -2, 0, 3$ respectively to get four simultaneous equations in $A_1, A_2, A_3, A_5$. When all is said and done, I get:

\begin{multline*}P(x) = \frac{17}{72(1-x)} + \frac{59}{288(1-x)^2} + \frac{1}{8(1-x)^3} \\ + \frac{1}{24(1-x)^4} + \frac{1}{8(1+x)} + \frac{1}{32(1+x)^2} + \frac{1}{16(1-ix)} \\ + \frac{1}{16(1+ix)} + \frac{1-\omega}{27(1-\omega x)} + \frac{1-\omega^2}{27(1-\omega^2 x)}\end{multline*}

And if I haven’t made a mistake, after turning each of these simple fractions into its own infinite series as follows:

\begin{align*}
\frac{1}{1-x} &= 1 + x + x^2 + \cdots \\
\frac{1}{(1-x)^2} &= 1 + 2x + 3x^2 + 4x^3 + \cdots \\
\frac{1}{(1-x)^3} &= 1 + \binom{3}{2}x + \binom{4}{2}x^2 + \binom{5}{2}x^3 + \cdots \\
\frac{1}{(1-x)^4} &= 1 + \binom{4}{3}x + \binom{5}{3}x^2 + \binom{6}{3}x^3 + \cdots \\
\frac{1}{1+x} &= 1 - x + x^2 - \cdots \\
\frac{1}{(1+x)^2} &= 1 - 2x + 3x^2 - 4x^3 + \cdots \\
\frac{1}{1+ix} &= 1 + ix - x^2 - ix^3 + x^4 + \cdots \\
\frac{1}{1-ix} &= 1 - ix - x^2 + ix^3 + x^4 - \cdots \\
\frac{1}{1-\omega^2 x} &= 1 + \omega^2 x + \omega x^2 + x^3 + \omega^2 x^4 + \omega x^5 + x^6 + \cdots \\
\frac{1}{1-\omega x} &= 1 + \omega x + \omega^2 x^2 + x^3 + \omega x^4 + \omega^2 x^5 + x^6 + \cdots
\end{align*}

And when we plug everything in, we get a coefficient for $a_n$ (reminder, this is the number of
solutions for partitions in four ordered natural numbers for $x=n+4$) of:
\begin{multline*}
a_n = \frac{17}{72} + \frac{59}{288}\binom{n+1}{1} + \frac{1}{8}\binom{n+2}{2} + \frac{1}{24}\binom{n+3}{3} \\
+ \frac{1}{8}(-1)^n + \frac{1}{32}\binom{n+1}{1}(-1)^n + \frac{1}{16}(i^n + (-i)^n) \\
+ \frac{1}{27}\left(\omega^{n} + \omega^{2n} - \omega^{n+1} - \omega^{2n+2}\right)
\end{multline*}

The $\frac{1}{16}(i^n + (-i)^n)$ terms equal 0 for odd terms, $\frac{1}{8}$ for terms
where $n$ is divisible by 4, and $-\frac{1}{8}$ for other even terms.

Similarly, the $\frac{1}{27}\left(\omega^{n} + \omega^{2n} - \omega^{n+1} - \omega^{2n+2}\right)$
terms equal zero, $-\frac{1}{9}$, or $\frac{1}{9}$, depending on whether $n$ has a remainder of
0, 1, or 2 when divided by 3.

Let’s check our closed form solution for $n=8$ - that is, for $x=12$. By inspection, we can see that the solutions are:

\begin{multline*}
(a_1,b_1,c_1,d_1) \in \{(2,0,0,0), (1,1,0,1), (1,0,2,0), (1,0,1,2),\\
(1,0,0,4), (0,2,1,0), (0,2,0,2), (0,1,2,1),(0,1,1,3), (0,1,0,5),\\
(0,0,4,0), (0,0,3,2), (0,0,2,4), (0,0,1,6), (0,0,0,8)\}
\end{multline*}

which gives 15 solutions. Again, the solution $(0,1,2,1)$ (for example) corresponds to $a_1 = 0, b_1=1, c_1=2, d_1=1$, which translates to $a'=0, b'=1, c'=3, d'=4$, or $a=1, b=2, c=4, d=5$.
Using our formula, we get:
\begin{multline*}
a_8 = \frac{17}{72} + \frac{59}{288}\binom{9}{1} + \frac{1}{8}\binom{10}{2} + \frac{1}{24}\binom{11}{3} \\ + \frac{1}{8}(-1)^8 + \frac{1}{32}\binom{9}{1}(-1)^8 + \frac{1}{16}(i^8 + (-i)^8) \\ + \frac{1}{27}\left(\omega^{8} + \omega^{16} - \omega^{9} - \omega^{18}\right)
\end{multline*}

\[a_8 = \frac{17}{72} + \frac{59}{32} + \frac{45}{8} + \frac{55}{8} + \frac{1}{8} + \frac{9}{32} + \frac{1}{8} - \frac{1}{9} = 15 \]

It’s pretty amazing that this complicated fractional expression including binomial coefficients works, but it does!

\section{Recurrence relation}

It is possible to calculate the number of partitions also using a recurrence relation. If we define:
$P_k(n)$ to be the number of ordered partitions of the number $n$ into exactly $k$ non-zero partitions, we can deduce the following:
\begin{itemize}
    \item $P_0(0) = 1$ (by definition - similar to defining $0! = 1$, this ensures the recurrence relationship below terminates in all cases).
    \item $P_k(n) = 0$ if $k\leq 0, n\leq 0$ and $k,n$ are not both $0$.
    \item $P_k(n) = P_k(n-k) + P_{k-1}(n-1)$ - that is, we have a choice to increment the size of all partitions by 1 and leave $n-k$ items to distribute across exactly $k$ buckets, or we can fix one bucket, and we have $n-1$ items to distribute across the other $k-1$ buckets.
\end{itemize}

We can come up with some quick short-cuts for $P_k(n)$ for small values of $k$:
\begin{itemize}
    \item $P_1(n) = 1$ for all $n \geq 0$ - that is, with exactly 1 partition, there is only one possible representation.
    \item $P_n(n) = 1$ is the transposed equivalent - with $n$ buckets and $n$ items, there is only one way to distribute the items so that no bucket is empty.
    \item $P_k(n) = 0$ if $k > 0, n< k$.
    \item $P_2(n) = \lfloor \frac{n}{2} \rfloor$
    \item $P_3(n) = \lfloor \frac{n-1}{2} \rfloor + \lfloor \frac{n-4}{2} \rfloor + \lfloor \frac{n-7}{2} \rfloor  + \cdots $
\end{itemize}

From this, we can reproduce the result above, albeit a little awkwardly:
\begin{align*}
    P_4(12) &= P_4(8) + P_3(11) \\
    &= P_4(4) + P_3(7) + P_3(8) + P_2(10) \\
    &= 1 + \lfloor \frac{6}{2} \rfloor + \lfloor \frac{3}{2} \rfloor + \lfloor \frac{7}{2} \rfloor + \lfloor \frac{4}{2} \rfloor + \lfloor \frac{10}{2} \rfloor \\
    &= 1 + 3 + 1 + 3 + 2 + 5 = 15    
\end{align*}

as before. I have not found any nice closed form solution to this recurrence relation, however, and the recurrence relation, while easy to calculate by computer, becomes very unwieldy when calculating by hand.

\end{document}
