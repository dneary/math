\documentclass{article}
\usepackage{latexsym}
\usepackage{amsmath}
\usepackage{amssymb}

\begin{document}
\title{Measure Theory notes}
\author{Dave Neary}

\maketitle

\section{Motivation for Lebesgue integral}

\subsection{Overview of the Riemann integral}

The Riemann integral of a continuous function $f:\mathbb{R} \rightarrow \mathbb{R}$ is defined as:

\[ \int_{a}^{b}f(x) dx = \lim_{n \rightarrow \infty}\sum_{i=1}^{n}f(x_{i}) \Delta x \]
 
where $x_{i} = a + (i-1)\Delta x$ and $\Delta x=\frac{b-a}{n}$

In other words, we partition the domain of the function into small slices, 
and calculate the area under the curve by multiplying the width of the slices 
by the value of the function at the beginning of the slice.

This works well for a certain class of functions, called Riemann-integrable
functions. These functions must satisfy the condition that the domain is $\mathbb{R}$,
and that limit above exists.

More generally, we can calculate an upper Riemann sum $U(f)$ by summing the 
areas using $\sup{f(x)}$ on each partition, and a lower sum $L(f)$ by using $\inf{f(x)}$
for each interval. The function $f$ is Riemann integrable when $\lim L(f) = \lim U(x)$. 

\subsection{Limitations of the Riemann integral}

It is possible to generalize the Riemann integral to two
or more dimensions, but the problem of finding an appropriate partition of the domain
means that for dimensions of the real numbers which are higher than $\mathbb{R}^2$,
the Riemann integral is limited.  In addition, we would like to consider other classes
of domains than the reals for functions - for example, probability spaces or generic
Hilbert spaces - where some alternative idea of the area under the curve (or more 
generally, the volume of a set) may make sense. Another limitation of the Riemann
integral is that there are useful classes of functions for which it does not converge,
but for which a reasonable value for the integral exists.
 
Another limitation of the Riemann integral is that there is only a very limited set of
functions for which  it is possible to say 
\[\int \sum_n f_n(x) dx = \sum_n \int f_n(x) dx \]

Namely, $f_n(x)$ must converge uniformly to $f(x)$, which is a very strong constraint.

As a result of these limitations, the idea of the Lebesgue integral is to partition
the function range instead of the domain. We then identify the subsets of the domain for
specific values of $f(x)$, and calculate their volume using a generic measure function
$\mu$. By taking finer and finer intervals of the range, we can get better and better 
estimates of the volume under the function with respect to the domain and the measure.

The remainder of this document will describe the characteristics of a domain, the
constraints required for a measure, which types of functions we can integrate, and
a precise definition of the Lebesgue integral. We will also include a selection of
proofs and problems which we can use the Lebesgue integral to solve.

\section{Measure spaces}

Working backwards, to define what we mean by an integrable function, we will need to
first define how to measure the volume of a subset of a domain (a measure), and to
define a measure, we must first define the types of sets which will be measurable.

Starting from a set $X$, a collection of subsets of $X$, $\mathcal{A}$, is called a
$\sigma$-algebra if it satisfies the following conditions:

\begin{itemize}
		\item $X \in \mathcal{A}$
		\item For each $A \in \mathcal{A}$, $X \setminus A \in \mathcal{A}$
		\item For a countable sequence of subsets $(A_n)_{n \in \mathbb{N}} \in \mathcal{A}$,
			\[\bigcup_{n} A_n \in \mathcal{A} \]
\end{itemize}

We will see when we define a measure why this is called a $\sigma$-algebra.

The pair $(X, \mathcal{A})$ is called a measure space.

\end{document}
