\documentclass{article}
\usepackage{latexsym}
\usepackage{amsmath}
\usepackage{amssymb}
\usepackage{graphicx}
\usepackage{amsthm}

\newtheorem{theorem}{Theorem}[section]
\newtheorem*{problem}{Problem}

\begin{document}
\title{Bernoulli Numbers}
\author{Dave Neary}

\maketitle

\section{Introduction}

I first came across the Bernoulli numbers in Faulhaber's formula for summing the $k$th 
powers of integers. Faulhaber's formula uses exponential generating functions to prove that:
\[ \sum_{i=1}^n i^k = \frac{1}{k+1} \sum_{j=0}^k \binom{k+1}{j} B_j n^{k+1-j} \]

where $\{B_j\}$ are the Bernoulli numbers. The first few Bernoulli numbers are:
\[ \{B_i\} = \{1, \frac{1}{2}, \frac{1}{6}, 0, -\frac{1}{30}, 0, \frac{1}{42}, \cdots\} \]

We can use this formula to show, for example, that the sum of the 4th powers of integers:
\begin{align*}
	\sum_{i=1}^n i^4 &= \frac{1}{5}\left( \binom{5}{0} B_0 n^{5} +  \binom{5}{1} B_1 n^{4} +  
	\binom{5}{2} B_2 n^{3} +  \binom{5}{3} B_3 n^{2} +  \binom{5}{4} B_4 n^{1} \right) \\
			 &= \frac{1}{5}\left( n^{5} +  \frac{5}{2} n^{4} +
			 \frac{5}{3} n^{3} -  \frac{1}{6} n  \right) \\
			 &= \frac{1}{5}n^5 + \frac{1}{2}n^4 + \frac{1}{3}n^3 - \frac{1}{30}n
\end{align*}

But where do these numbers come from, what other applications do they have, and why are they
important enough to have been given a name?

\section{Bernoulli polynomials}

\end{document}
