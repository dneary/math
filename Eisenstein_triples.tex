\documentclass{article}
\usepackage{latexsym}
\usepackage{amsmath}
\usepackage{amssymb}
\usepackage{graphicx}
\usepackage{amsthm}

\newtheorem{theorem}{Theorem}[section]
\newtheorem*{problem}{Problem}

\begin{document}

\title{Eisenstein triples}
\author{Dave Neary}

\maketitle

\section{Introduction}

\begin{problem}
	Find positive integer solutions to the equation:
	\[ \frac{1}{a+b} + \frac{1}{b+c} = \frac{3}{a+b+c} \]
\end{problem}

Cross-multiplying by $a+b+c$:
\[ \frac{a+b+c}{a+b} + \frac{a+b+c}{b+c} = 3 \]

Then we can simplify:
\begin{align*}
	\frac{a+b}{a+b} + \frac{c}{a+b} + \frac{a}{b+c} + \frac{b+c}{b+c} &= 3 \\
	\frac{c}{a+b} + \frac{a}{b+c} &= 1 \\
	c(b+c) + a(a+b) &= (a+b)(b+c) \\
	a^2 -ac +c^2 &= b^2
\end{align*}

\section{Geometric approach}

The cosine rule for triangles is:
\[ b^2 = a^2 + c^2 - 2ac \cos B \]

The equation above is simply the cosine rule for triangles, applied with the angle 
$B = \frac{\pi}{3}$ radians. So the lengths $a, b, c$ are side lengths of a triangle with 
an angle of $\frac{\pi}{3}$ radians (or 60 degrees) at angle $B$. We could try some 
geometric approaches at this point to try to find a general formula for integer lengths.

One approach which can be used to find Pythagorean triples (that is, integers which are 
the sides of a right-angled triangle) is to consider the unit circle:
\[ x^2 + y^2 = 1 \]
and to search for rational points on this circle. Then, if we find points
$(\frac{a}{c},\frac{b}{c})$ on the circle, we can generate a Pythagorean triple $(a,b,c)$.

And one way to do that is to consider the lines through the point $(-1,0)$ with a rational
slope. For every rational number, this will intersect the unit circile in two points, and we
can show that if the slope is rational, then the intersection points will also be rational.
That is, given the equation $y = m(x+1)$, and the unit circle $x^2+y^2=1$, the intersection
points satisfy:
\begin{align*}
	x^2+m^2(x+1)^2 &= 1 \\
	(1+m^2)x^2 +2m^2x +(m^2-1) &= 0 \\
	x &= \frac{-2m^2 \pm \sqrt{4m^4 -4(m^2+1)(m^2-1)}}{2(1+m^2)} \\
	x &= \frac{-m^2 \pm 1}{m^2+1} = -1 \text{ or } \frac{1-m^2}{1+m^2} \\
	y &= 0 \text{ or } \frac{2m}{1+m^2} 
\end{align*}

Which results in the familiar formula for generating Pythagoream triples $(m^2-n^2, 2mn, m^2+n^2)$.

We can try a similar approach here with the ellipse $x^2-xy+y^2=1$. The line through the point 
$(-1,0)$ will again be $y = m(x+1)$ and we can solve for the intersection points by substitution:

\begin{align*}
	x^2 - x(m(x+1)) + (m(x+1))^2 &= 1 \\
	(1-m+m^2)x^2 +(2m^2-m)x +(m^2-1) &= 0 \\
	x &= \frac{-(2m^2-m) \pm \sqrt{(2m^2-m)^2 -4(m^2-1)(1-m+m^2)}}{2(1-m+m^2)} \\
	x &= \frac{-2m^2 +m \pm (m-2)}{2(m^2-m+1)} = -1 \text{ or } \frac{1-m^2}{1-m+m^2} \\
	y &= 0 \text{ or } \frac{2m-m^2}{1-m+m^2} 
\end{align*}

Which means, given any rational number $m=\frac{a}{b}$ for which the line $y=m(x+1)$ intersects
the ellipse in two places, we can find a rational point $(x,y)$ which will satisfy the equation
of the ellipse.

Let's check that it works: we will try $m=\frac{2}{5}$. Then:
\[ (x,y) = \left(\frac{1 - (\frac{2}{5})^2}{1-\frac{2}{5} + (\frac{2}{5})^2}, 
\frac{2(\frac{2}{5}) - (\frac{2}{5})^2}{1-\frac{2}{5} + (\frac{2}{5})^2} \right)\]
Multiplying above and below the line of both by $5^2$, we get:

\[ (x,y) = (\frac{5^2 - 2^2}{5^2- 2\times5 + 2^2}, 
\frac{2\times2\times5 - 2^2}{5^2 - 2\times5 + 2^2}) = (\frac{21}{19},\frac{16}{19}) \]

And we can check that $19^2 = 361 = 21^2 - 21\times 16 + 16^2 = 441-336+256$ gives us a triple
satisfying the equation. We can repeat this for any rational number $\frac{m}{n}$ giving a triple
$(a,b,c) = (n^2-m^2, 2mn - m^2, n^2-nm+m^2)$ with $a^2 - ab + b^2 = c^2$.

\section{Eisenstein integers approach}

By factoring $a^2 + b^2 = (a+ib)(a-ib)$ we can also generate Pythagorean triples. Defining 
$N(a+ib) = a^2 + b^2$, we can show that $N(z_1 z_2) = N(z_1)N(z_2)$, and therefore, that:
\begin{align*}
	(m^2+n^2)^2 & = (N(m+in))^2 \\
	&= N((m+in)^2) \\
	&= N(m^2-n^2 + 2mni) \\
	&= (m^2-n^2)^2 + (2mn)^2
\end{align*}

which gives the Pythagorean triple $(a,b,c) = (m^2-n^2,2mn,m^2+n^2)$.

Inspired by this method of finding Pythagorean triples by squaring Gaussian integers, let's
see if there is an analog for Eisenstein triples too. It is not a surprise to learn that there is,
and that such complex numbers are known as Eisenstein integers.

Starting from the equation $ a^2 - ab + b^2 = c^2$, we can complete the square, and factorize
across the complex numbers, and see what happens.

\begin{align*}
	a^2 - ab + b^2 &= \left(a-\frac{b}{2}\right)^2 + \frac{3}{4}b^2 \\
	&= \left(a-\frac{b}{2}\right)^2 + \left(\frac{\sqrt{3}b}{2}\right) \\
	&= \left(a + b\frac{-1-\sqrt{3}}{2}\right)\left(a + b\frac{-1+\sqrt{3}}{2}\right) \\
	&= \left(a + \omega b\right) \left(a + \omega^2 b\right)
\end{align*}
where $\omega$ is the primitive cube root of 1.

Some results about the cube roots of 1:
\[ 1 + \omega + \omega^2 = 0 \]
\[ \omega^2 = \bar{\omega} = -1 - \omega\]
\[ \omega^3 = 1 = \omega \bar{\omega} \]

Eisenstein integers are any numbers that can be written in the form $a+\omega b$ for
$a,b \in \mathbb{Z}$. They form a commutative ring (which means that they are closed under addition,
multiplication, and both multiplication and addition operations are commutative and distributive, 
and contain a multiplicative and additive identity).

What makes them super useful for our purposes here is that we now have a way to express $c^2$ as a
product of eisenstein integers:
\[ c^2 = (a+\omega b) (a + \bar{\omega} b) \]

And we can use the same norm argument that worked for Pythagorean triples here, if we define
$N(m+\omega n) = m^2-mn+n^2$ (which is the standard complex number norm). Then:

\begin{align*}
	N((m+\omega n)^2) &= N(m^2+2mn\omega + n^2(-1-\omega)) \\
	&= N(m^2-n^2+(2mn-n^2)\omega) \\
	&= (m^2-n^2)^2 - (m^2-n^2)(2mn-n^2) + (2mn-n^2)^2 \\
	N((m+\omega n)^2) &= (N(m+\omega n))^2 \\
	&= (m^2 - mn + n^2)
\end{align*}

So we have found another method to prove that given $m,n \in \mathbb{Z}$ we can construct a triple
$(a,b,c) = (m^2-n^2, 2mn-n^2, m+2-mn+n^2)$ such that $a^2-ab+b^2 = c^2$.

\end{document}
